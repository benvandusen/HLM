\documentclass[]{article}
\usepackage{lmodern}
\usepackage{amssymb,amsmath}
\usepackage{ifxetex,ifluatex}
\usepackage{fixltx2e} % provides \textsubscript
\ifnum 0\ifxetex 1\fi\ifluatex 1\fi=0 % if pdftex
  \usepackage[T1]{fontenc}
  \usepackage[utf8]{inputenc}
\else % if luatex or xelatex
  \ifxetex
    \usepackage{mathspec}
  \else
    \usepackage{fontspec}
  \fi
  \defaultfontfeatures{Ligatures=TeX,Scale=MatchLowercase}
\fi
% use upquote if available, for straight quotes in verbatim environments
\IfFileExists{upquote.sty}{\usepackage{upquote}}{}
% use microtype if available
\IfFileExists{microtype.sty}{%
\usepackage{microtype}
\UseMicrotypeSet[protrusion]{basicmath} % disable protrusion for tt fonts
}{}
\usepackage[margin=1in]{geometry}
\usepackage{hyperref}
\hypersetup{unicode=true,
            pdftitle={Supplemental Materia},
            pdfborder={0 0 0},
            breaklinks=true}
\urlstyle{same}  % don't use monospace font for urls
\usepackage{color}
\usepackage{fancyvrb}
\newcommand{\VerbBar}{|}
\newcommand{\VERB}{\Verb[commandchars=\\\{\}]}
\DefineVerbatimEnvironment{Highlighting}{Verbatim}{commandchars=\\\{\}}
% Add ',fontsize=\small' for more characters per line
\usepackage{framed}
\definecolor{shadecolor}{RGB}{248,248,248}
\newenvironment{Shaded}{\begin{snugshade}}{\end{snugshade}}
\newcommand{\KeywordTok}[1]{\textcolor[rgb]{0.13,0.29,0.53}{\textbf{#1}}}
\newcommand{\DataTypeTok}[1]{\textcolor[rgb]{0.13,0.29,0.53}{#1}}
\newcommand{\DecValTok}[1]{\textcolor[rgb]{0.00,0.00,0.81}{#1}}
\newcommand{\BaseNTok}[1]{\textcolor[rgb]{0.00,0.00,0.81}{#1}}
\newcommand{\FloatTok}[1]{\textcolor[rgb]{0.00,0.00,0.81}{#1}}
\newcommand{\ConstantTok}[1]{\textcolor[rgb]{0.00,0.00,0.00}{#1}}
\newcommand{\CharTok}[1]{\textcolor[rgb]{0.31,0.60,0.02}{#1}}
\newcommand{\SpecialCharTok}[1]{\textcolor[rgb]{0.00,0.00,0.00}{#1}}
\newcommand{\StringTok}[1]{\textcolor[rgb]{0.31,0.60,0.02}{#1}}
\newcommand{\VerbatimStringTok}[1]{\textcolor[rgb]{0.31,0.60,0.02}{#1}}
\newcommand{\SpecialStringTok}[1]{\textcolor[rgb]{0.31,0.60,0.02}{#1}}
\newcommand{\ImportTok}[1]{#1}
\newcommand{\CommentTok}[1]{\textcolor[rgb]{0.56,0.35,0.01}{\textit{#1}}}
\newcommand{\DocumentationTok}[1]{\textcolor[rgb]{0.56,0.35,0.01}{\textbf{\textit{#1}}}}
\newcommand{\AnnotationTok}[1]{\textcolor[rgb]{0.56,0.35,0.01}{\textbf{\textit{#1}}}}
\newcommand{\CommentVarTok}[1]{\textcolor[rgb]{0.56,0.35,0.01}{\textbf{\textit{#1}}}}
\newcommand{\OtherTok}[1]{\textcolor[rgb]{0.56,0.35,0.01}{#1}}
\newcommand{\FunctionTok}[1]{\textcolor[rgb]{0.00,0.00,0.00}{#1}}
\newcommand{\VariableTok}[1]{\textcolor[rgb]{0.00,0.00,0.00}{#1}}
\newcommand{\ControlFlowTok}[1]{\textcolor[rgb]{0.13,0.29,0.53}{\textbf{#1}}}
\newcommand{\OperatorTok}[1]{\textcolor[rgb]{0.81,0.36,0.00}{\textbf{#1}}}
\newcommand{\BuiltInTok}[1]{#1}
\newcommand{\ExtensionTok}[1]{#1}
\newcommand{\PreprocessorTok}[1]{\textcolor[rgb]{0.56,0.35,0.01}{\textit{#1}}}
\newcommand{\AttributeTok}[1]{\textcolor[rgb]{0.77,0.63,0.00}{#1}}
\newcommand{\RegionMarkerTok}[1]{#1}
\newcommand{\InformationTok}[1]{\textcolor[rgb]{0.56,0.35,0.01}{\textbf{\textit{#1}}}}
\newcommand{\WarningTok}[1]{\textcolor[rgb]{0.56,0.35,0.01}{\textbf{\textit{#1}}}}
\newcommand{\AlertTok}[1]{\textcolor[rgb]{0.94,0.16,0.16}{#1}}
\newcommand{\ErrorTok}[1]{\textcolor[rgb]{0.64,0.00,0.00}{\textbf{#1}}}
\newcommand{\NormalTok}[1]{#1}
\usepackage{graphicx,grffile}
\makeatletter
\def\maxwidth{\ifdim\Gin@nat@width>\linewidth\linewidth\else\Gin@nat@width\fi}
\def\maxheight{\ifdim\Gin@nat@height>\textheight\textheight\else\Gin@nat@height\fi}
\makeatother
% Scale images if necessary, so that they will not overflow the page
% margins by default, and it is still possible to overwrite the defaults
% using explicit options in \includegraphics[width, height, ...]{}
\setkeys{Gin}{width=\maxwidth,height=\maxheight,keepaspectratio}
\IfFileExists{parskip.sty}{%
\usepackage{parskip}
}{% else
\setlength{\parindent}{0pt}
\setlength{\parskip}{6pt plus 2pt minus 1pt}
}
\setlength{\emergencystretch}{3em}  % prevent overfull lines
\providecommand{\tightlist}{%
  \setlength{\itemsep}{0pt}\setlength{\parskip}{0pt}}
\setcounter{secnumdepth}{0}
% Redefines (sub)paragraphs to behave more like sections
\ifx\paragraph\undefined\else
\let\oldparagraph\paragraph
\renewcommand{\paragraph}[1]{\oldparagraph{#1}\mbox{}}
\fi
\ifx\subparagraph\undefined\else
\let\oldsubparagraph\subparagraph
\renewcommand{\subparagraph}[1]{\oldsubparagraph{#1}\mbox{}}
\fi

%%% Use protect on footnotes to avoid problems with footnotes in titles
\let\rmarkdownfootnote\footnote%
\def\footnote{\protect\rmarkdownfootnote}

%%% Change title format to be more compact
\usepackage{titling}

% Create subtitle command for use in maketitle
\newcommand{\subtitle}[1]{
  \posttitle{
    \begin{center}\large#1\end{center}
    }
}

\setlength{\droptitle}{-2em}

  \title{Supplemental Materia}
    \pretitle{\vspace{\droptitle}\centering\huge}
  \posttitle{\par}
    \author{}
    \preauthor{}\postauthor{}
    \date{}
    \predate{}\postdate{}
  

\begin{document}
\maketitle

Load data

\begin{Shaded}
\begin{Highlighting}[]
\KeywordTok{load}\NormalTok{(}\StringTok{"~/HLM/Sample_data.Rda"}\NormalTok{) }\CommentTok{#note, you will need to insert the path to the data file here}
\end{Highlighting}
\end{Shaded}

Load packages

\begin{Shaded}
\begin{Highlighting}[]
\KeywordTok{library}\NormalTok{(tidyr)}
\KeywordTok{library}\NormalTok{(lme4)}
\KeywordTok{library}\NormalTok{(multcomp)}
\KeywordTok{library}\NormalTok{(Matrix)}
\KeywordTok{library}\NormalTok{(ggplot2)}
\KeywordTok{library}\NormalTok{(lattice)}
\KeywordTok{library}\NormalTok{(stringr)}
\KeywordTok{library}\NormalTok{(dplyr) }
\end{Highlighting}
\end{Shaded}

Creating new variables

\begin{Shaded}
\begin{Highlighting}[]
\NormalTok{  class_means <-}\StringTok{ }\NormalTok{SampleData }\OperatorTok\StringTok{ }\KeywordTok{group_by}\NormalTok{(crse_id) }\OperatorTok\StringTok{ }\KeywordTok{summarise}\NormalTok{(}\DataTypeTok{pre_mean_class =} \KeywordTok{mean}\NormalTok{(pre_scor)) }\CommentTok{# Creates a new dataframe with course means}
\NormalTok{  class_means}\OperatorTok{$}\NormalTok{class_pre_cent <-}\StringTok{ }\NormalTok{class_means}\OperatorTok{$}\NormalTok{pre_mean_class }\OperatorTok{-}\StringTok{ }\KeywordTok{mean}\NormalTok{(class_means}\OperatorTok{$}\NormalTok{pre_mean_class) }\CommentTok{# Grand centers the course means}
\NormalTok{  SampleData <-}\StringTok{ }\KeywordTok{left_join}\NormalTok{(SampleData,class_means, }\DataTypeTok{by=}\StringTok{"crse_id"}\NormalTok{) }\CommentTok{# adds the course means back into the sample data frame}
\NormalTok{  SampleData}\OperatorTok{$}\NormalTok{stud_pre_cent <-}\StringTok{ }\NormalTok{SampleData}\OperatorTok{$}\NormalTok{pre_scor }\OperatorTok{-}\StringTok{ }\NormalTok{SampleData}\OperatorTok{$}\NormalTok{pre_mean_class }\CommentTok{# Group centers student prescores about their course means}
\NormalTok{  SampleData}\OperatorTok{$}\NormalTok{stud_pre_grand <-}\StringTok{ }\NormalTok{SampleData}\OperatorTok{$}\NormalTok{pre_scor}\OperatorTok{-}\KeywordTok{mean}\NormalTok{(SampleData}\OperatorTok{$}\NormalTok{pre_scor)  }\CommentTok{# Grand centers student prescores about the grand mean of all student prescores}
\NormalTok{  SampleData}\OperatorTok{$}\NormalTok{gain <-}\StringTok{ }\NormalTok{SampleData}\OperatorTok{$}\NormalTok{pst_scor }\OperatorTok{-}\StringTok{ }\NormalTok{SampleData}\OperatorTok{$}\NormalTok{pre_scor }\CommentTok{# calculates the gain}
\NormalTok{  SampleData}\OperatorTok{$}\NormalTok{collabnla <-}\StringTok{ }\KeywordTok{ifelse}\NormalTok{(SampleData}\OperatorTok{$}\NormalTok{colablrn}\OperatorTok{==}\DecValTok{1}\NormalTok{,}\KeywordTok{ifelse}\NormalTok{(SampleData}\OperatorTok{$}\NormalTok{used_las}\OperatorTok{==}\DecValTok{0}\NormalTok{,}\DecValTok{1}\NormalTok{,}\DecValTok{0}\NormalTok{),}\DecValTok{0}\NormalTok{) }\CommentTok{# Creates a dummy variable for whether courses used collaborative learning without LAs}
\end{Highlighting}
\end{Shaded}

Calculating the descriptive statistics

\begin{Shaded}
\begin{Highlighting}[]
\CommentTok{#Make one categorical variable with all three types of instruction}
\NormalTok{  SampleData}\OperatorTok{$}\NormalTok{instruction <-}\StringTok{ }\KeywordTok{ifelse}\NormalTok{(SampleData}\OperatorTok{$}\NormalTok{used_las}\OperatorTok{==}\DecValTok{1}\NormalTok{,}\StringTok{"Used_LAs"}\NormalTok{,}\KeywordTok{ifelse}\NormalTok{(SampleData}\OperatorTok{$}\NormalTok{collabnla}\OperatorTok{==}\DecValTok{1}\NormalTok{,}\StringTok{"Collab_No_LA"}\NormalTok{,}\StringTok{"Lecture"}\NormalTok{))}

\CommentTok{# Make data frame of student means by instruction type (disaggrgation)}
\NormalTok{  student_means <-}\StringTok{ }\NormalTok{SampleData }\OperatorTok\StringTok{ }\KeywordTok{group_by}\NormalTok{(instruction) }\OperatorTok\StringTok{ }\KeywordTok{summarise}\NormalTok{(}\DataTypeTok{mean_gain =} \KeywordTok{mean}\NormalTok{(gain))}
\NormalTok{  return <-}\StringTok{ }\NormalTok{student_means}

\CommentTok{#Make a data frame of course means by instruction type (aggregation)}
\NormalTok{  class_means <-}\StringTok{ }\NormalTok{SampleData }\OperatorTok\StringTok{ }\KeywordTok{group_by}\NormalTok{(crse_id) }\OperatorTok\StringTok{ }\KeywordTok{summarise}\NormalTok{(}\DataTypeTok{gain =} \KeywordTok{mean}\NormalTok{(gain))}
\NormalTok{  class_means <-}\StringTok{ }\KeywordTok{left_join}\NormalTok{(class_means,}\KeywordTok{unique}\NormalTok{(SampleData[}\KeywordTok{c}\NormalTok{(}\DecValTok{3}\NormalTok{,}\DecValTok{13}\NormalTok{)]), }\DataTypeTok{by =} \StringTok{"crse_id"}\NormalTok{) }\CommentTok{#need to replace these column numbers}
\NormalTok{  class_means <-}\StringTok{ }\NormalTok{class_means }\OperatorTok\StringTok{ }\KeywordTok{group_by}\NormalTok{(instruction) }\OperatorTok\StringTok{ }\KeywordTok{summarise}\NormalTok{(}\DataTypeTok{gain =} \KeywordTok{mean}\NormalTok{(gain))}
\NormalTok{  return <-}\StringTok{ }\NormalTok{class_means}
\end{Highlighting}
\end{Shaded}

Define models (We ultimately used Model 3 as our simplest model that
explained the most variance)

\begin{Shaded}
\begin{Highlighting}[]
\CommentTok{#HLM models}
\NormalTok{hlm_mod1 <-}\StringTok{ }\NormalTok{(gain }\OperatorTok{~}\StringTok{ }\DecValTok{1} \OperatorTok{+}\StringTok{ }\NormalTok{(}\DecValTok{1}\OperatorTok{|}\NormalTok{crse_id))}
\NormalTok{hlm_mod2 <-}\StringTok{ }\NormalTok{(gain }\OperatorTok{~}\StringTok{ }\DecValTok{1} \OperatorTok{+}\StringTok{ }\NormalTok{used_las }\OperatorTok{+}\StringTok{ }\NormalTok{collabnla }\OperatorTok{+}\StringTok{ }\NormalTok{(}\DecValTok{1}\OperatorTok{|}\NormalTok{crse_id))}
\NormalTok{hlm_mod3 <-}\StringTok{ }\NormalTok{(gain }\OperatorTok{~}\StringTok{ }\DecValTok{1} \OperatorTok{+}\StringTok{ }\NormalTok{stud_pre_cent }\OperatorTok{+}\StringTok{  }\NormalTok{used_las }\OperatorTok{+}\StringTok{ }\NormalTok{collabnla }\OperatorTok{+}\StringTok{ }\NormalTok{(}\DecValTok{1}\OperatorTok{|}\NormalTok{crse_id)) }
\NormalTok{hlm_mod4 <-}\StringTok{ }\NormalTok{(gain }\OperatorTok{~}\StringTok{ }\DecValTok{1} \OperatorTok{+}\StringTok{ }\NormalTok{stud_pre_cent }\OperatorTok{+}\StringTok{  }\NormalTok{used_las }\OperatorTok{+}\StringTok{ }\NormalTok{collabnla }\OperatorTok{+}\StringTok{ }\NormalTok{(}\DecValTok{1}\OperatorTok{+}\StringTok{ }\NormalTok{stud_pre_cent}\OperatorTok{|}\NormalTok{crse_id))}
\NormalTok{hlm_mod5 <-}\StringTok{ }\NormalTok{(gain }\OperatorTok{~}\StringTok{ }\DecValTok{1} \OperatorTok{+}\StringTok{ }\NormalTok{stud_pre_cent }\OperatorTok{+}\StringTok{  }\NormalTok{used_las }\OperatorTok{+}\StringTok{ }\NormalTok{collabnla }\OperatorTok{+}\StringTok{ }\NormalTok{class_pre_cent }\OperatorTok{+}\StringTok{ }\NormalTok{(}\DecValTok{1}\OperatorTok{|}\NormalTok{crse_id))}
\NormalTok{hlm_mod6 <-}\StringTok{ }\NormalTok{(gain }\OperatorTok{~}\StringTok{ }\DecValTok{1} \OperatorTok{+}\StringTok{ }\NormalTok{stud_pre_cent }\OperatorTok{+}\StringTok{  }\NormalTok{used_las }\OperatorTok{+}\StringTok{ }\NormalTok{collabnla }\OperatorTok{+}\StringTok{ }\NormalTok{FMCE }\OperatorTok{+}\StringTok{ }\NormalTok{(}\DecValTok{1}\OperatorTok{|}\NormalTok{crse_id))}

\CommentTok{#MLR models}
\NormalTok{mlr_mod1 <-}\StringTok{ }\NormalTok{(gain }\OperatorTok{~}\StringTok{ }\DecValTok{1}\NormalTok{)}
\NormalTok{mlr_mod2 <-}\StringTok{ }\NormalTok{(gain }\OperatorTok{~}\StringTok{ }\DecValTok{1} \OperatorTok{+}\StringTok{ }\NormalTok{used_las }\OperatorTok{+}\StringTok{ }\NormalTok{collabnla)}
\NormalTok{mlr_mod3 <-}\StringTok{ }\NormalTok{(gain }\OperatorTok{~}\StringTok{ }\DecValTok{1} \OperatorTok{+}\StringTok{ }\NormalTok{stud_pre_cent }\OperatorTok{+}\StringTok{  }\NormalTok{used_las }\OperatorTok{+}\StringTok{ }\NormalTok{collabnla)}
\NormalTok{mlr_mod4 <-}\StringTok{ }\NormalTok{(gain }\OperatorTok{~}\StringTok{ }\DecValTok{1} \OperatorTok{+}\StringTok{ }\NormalTok{stud_pre_cent }\OperatorTok{+}\StringTok{  }\NormalTok{used_las }\OperatorTok{+}\StringTok{ }\NormalTok{collabnla)}
\NormalTok{mlr_mod5 <-}\StringTok{ }\NormalTok{(gain }\OperatorTok{~}\StringTok{ }\DecValTok{1} \OperatorTok{+}\StringTok{ }\NormalTok{stud_pre_cent }\OperatorTok{+}\StringTok{  }\NormalTok{used_las }\OperatorTok{+}\StringTok{ }\NormalTok{collabnla }\OperatorTok{+}\StringTok{ }\NormalTok{class_pre_cent)}
\NormalTok{mlr_mod6 <-}\StringTok{ }\NormalTok{(gain }\OperatorTok{~}\StringTok{ }\DecValTok{1} \OperatorTok{+}\StringTok{ }\NormalTok{stud_pre_cent }\OperatorTok{+}\StringTok{  }\NormalTok{used_las }\OperatorTok{+}\StringTok{ }\NormalTok{collabnla }\OperatorTok{+}\StringTok{ }\NormalTok{FMCE)}
\end{Highlighting}
\end{Shaded}

Run models

\begin{Shaded}
\begin{Highlighting}[]
\CommentTok{#HLM models}
\NormalTok{HLM1 <-}\StringTok{ }\KeywordTok{lmer}\NormalTok{(hlm_mod1, }\DataTypeTok{data=}\NormalTok{SampleData)}
\NormalTok{HLM2 <-}\StringTok{ }\KeywordTok{lmer}\NormalTok{(hlm_mod2, }\DataTypeTok{data=}\NormalTok{SampleData)}
\NormalTok{HLM3 <-}\StringTok{ }\KeywordTok{lmer}\NormalTok{(hlm_mod3, }\DataTypeTok{data=}\NormalTok{SampleData)}
\NormalTok{HLM4 <-}\StringTok{ }\KeywordTok{lmer}\NormalTok{(hlm_mod4, }\DataTypeTok{data=}\NormalTok{SampleData)}
\NormalTok{HLM5 <-}\StringTok{ }\KeywordTok{lmer}\NormalTok{(hlm_mod5, }\DataTypeTok{data=}\NormalTok{SampleData)}
\NormalTok{HLM6 <-}\StringTok{ }\KeywordTok{lmer}\NormalTok{(hlm_mod6, }\DataTypeTok{data=}\NormalTok{SampleData)}

\CommentTok{#MLR models}
\NormalTok{MLR1 <-}\StringTok{ }\KeywordTok{lm}\NormalTok{(mlr_mod1, }\DataTypeTok{data=}\NormalTok{SampleData)}
\NormalTok{MLR2 <-}\StringTok{ }\KeywordTok{lm}\NormalTok{(mlr_mod2, }\DataTypeTok{data=}\NormalTok{SampleData)}
\NormalTok{MLR3 <-}\StringTok{ }\KeywordTok{lm}\NormalTok{(mlr_mod3, }\DataTypeTok{data=}\NormalTok{SampleData)}
\NormalTok{MLR4 <-}\StringTok{ }\KeywordTok{lm}\NormalTok{(mlr_mod4, }\DataTypeTok{data=}\NormalTok{SampleData)}
\NormalTok{MLR5 <-}\StringTok{ }\KeywordTok{lm}\NormalTok{(mlr_mod5, }\DataTypeTok{data=}\NormalTok{SampleData)}
\NormalTok{MLR6 <-}\StringTok{ }\KeywordTok{lm}\NormalTok{(mlr_mod6, }\DataTypeTok{data=}\NormalTok{SampleData)}
\end{Highlighting}
\end{Shaded}

Model outputs

\begin{Shaded}
\begin{Highlighting}[]
\CommentTok{#HLM models}
\KeywordTok{summary}\NormalTok{(HLM1)}
\end{Highlighting}
\end{Shaded}

\begin{verbatim}
## Linear mixed model fit by REML ['lmerMod']
## Formula: gain ~ 1 + (1 | crse_id)
##    Data: SampleData
## 
## REML criterion at convergence: 52992.6
## 
## Scaled residuals: 
##     Min      1Q  Median      3Q     Max 
## -4.8389 -0.6228 -0.0285  0.6071  3.8036 
## 
## Random effects:
##  Groups   Name        Variance Std.Dev.
##  crse_id  (Intercept)  63.01    7.938  
##  Residual             411.31   20.281  
## Number of obs: 5959, groups:  crse_id, 112
## 
## Fixed effects:
##             Estimate Std. Error t value
## (Intercept)  18.4317     0.8332   22.12
\end{verbatim}

\begin{Shaded}
\begin{Highlighting}[]
\KeywordTok{summary}\NormalTok{(HLM2)}
\end{Highlighting}
\end{Shaded}

\begin{verbatim}
## Linear mixed model fit by REML ['lmerMod']
## Formula: gain ~ 1 + used_las + collabnla + (1 | crse_id)
##    Data: SampleData
## 
## REML criterion at convergence: 52976.9
## 
## Scaled residuals: 
##     Min      1Q  Median      3Q     Max 
## -4.8423 -0.6224 -0.0272  0.6050  3.8016 
## 
## Random effects:
##  Groups   Name        Variance Std.Dev.
##  crse_id  (Intercept)  58.36    7.64   
##  Residual             411.31   20.28   
## Number of obs: 5959, groups:  crse_id, 112
## 
## Fixed effects:
##             Estimate Std. Error t value
## (Intercept)   14.090      2.048   6.880
## used_las       6.080      2.279   2.668
## collabnla      1.997      2.755   0.725
## 
## Correlation of Fixed Effects:
##           (Intr) usd_ls
## used_las  -0.899       
## collabnla -0.743  0.668
\end{verbatim}

\begin{Shaded}
\begin{Highlighting}[]
\KeywordTok{summary}\NormalTok{(HLM3)}
\end{Highlighting}
\end{Shaded}

\begin{verbatim}
## Linear mixed model fit by REML ['lmerMod']
## Formula: gain ~ 1 + stud_pre_cent + used_las + collabnla + (1 | crse_id)
##    Data: SampleData
## 
## REML criterion at convergence: 51894.8
## 
## Scaled residuals: 
##     Min      1Q  Median      3Q     Max 
## -4.4168 -0.6587 -0.0103  0.6496  3.7686 
## 
## Random effects:
##  Groups   Name        Variance Std.Dev.
##  crse_id  (Intercept)  60.87    7.802  
##  Residual             341.47   18.479  
## Number of obs: 5959, groups:  crse_id, 112
## 
## Fixed effects:
##               Estimate Std. Error t value
## (Intercept)   14.19542    2.05732   6.900
## stud_pre_cent -0.45137    0.01305 -34.595
## used_las       5.97094    2.28891   2.609
## collabnla      1.69127    2.75784   0.613
## 
## Correlation of Fixed Effects:
##             (Intr) std_p_ usd_ls
## stud_pr_cnt  0.000              
## used_las    -0.899  0.000       
## collabnla   -0.746  0.000  0.671
\end{verbatim}

\begin{Shaded}
\begin{Highlighting}[]
\KeywordTok{summary}\NormalTok{(HLM4)}
\end{Highlighting}
\end{Shaded}

\begin{verbatim}
## Linear mixed model fit by REML ['lmerMod']
## Formula: 
## gain ~ 1 + stud_pre_cent + used_las + collabnla + (1 + stud_pre_cent |  
##     crse_id)
##    Data: SampleData
## 
## REML criterion at convergence: 51861.3
## 
## Scaled residuals: 
##     Min      1Q  Median      3Q     Max 
## -4.3551 -0.6516 -0.0026  0.6422  3.8390 
## 
## Random effects:
##  Groups   Name          Variance  Std.Dev. Corr 
##  crse_id  (Intercept)    61.25363  7.8265       
##           stud_pre_cent   0.01174  0.1083  -0.61
##  Residual               337.58933 18.3736       
## Number of obs: 5959, groups:  crse_id, 112
## 
## Fixed effects:
##               Estimate Std. Error t value
## (Intercept)   13.54778    1.98378   6.829
## stud_pre_cent -0.43741    0.01826 -23.950
## used_las       7.17291    2.18797   3.278
## collabnla      1.02653    2.63541   0.390
## 
## Correlation of Fixed Effects:
##             (Intr) std_p_ usd_ls
## stud_pr_cnt -0.102              
## used_las    -0.893 -0.045       
## collabnla   -0.743 -0.014  0.676
## convergence code: 0
## Model is nearly unidentifiable: very large eigenvalue
##  - Rescale variables?
\end{verbatim}

\begin{Shaded}
\begin{Highlighting}[]
\KeywordTok{summary}\NormalTok{(HLM5)}
\end{Highlighting}
\end{Shaded}

\begin{verbatim}
## Linear mixed model fit by REML ['lmerMod']
## Formula: 
## gain ~ 1 + stud_pre_cent + used_las + collabnla + class_pre_cent +  
##     (1 | crse_id)
##    Data: SampleData
## 
## REML criterion at convergence: 51897.7
## 
## Scaled residuals: 
##     Min      1Q  Median      3Q     Max 
## -4.4171 -0.6588 -0.0109  0.6487  3.7768 
## 
## Random effects:
##  Groups   Name        Variance Std.Dev.
##  crse_id  (Intercept)  61.32    7.831  
##  Residual             341.49   18.479  
## Number of obs: 5959, groups:  crse_id, 112
## 
## Fixed effects:
##                Estimate Std. Error t value
## (Intercept)    14.37811    2.16418   6.644
## stud_pre_cent  -0.45137    0.01305 -34.594
## used_las        5.78287    2.39202   2.418
## collabnla       1.42664    2.91740   0.489
## class_pre_cent  0.02489    0.09034   0.276
## 
## Correlation of Fixed Effects:
##             (Intr) std_p_ usd_ls cllbnl
## stud_pr_cnt  0.000                     
## used_las    -0.907  0.000              
## collabnla   -0.770  0.000  0.699       
## clss_pr_cnt  0.301  0.000 -0.280 -0.318
\end{verbatim}

\begin{Shaded}
\begin{Highlighting}[]
\KeywordTok{summary}\NormalTok{(HLM6)}
\end{Highlighting}
\end{Shaded}

\begin{verbatim}
## Linear mixed model fit by REML ['lmerMod']
## Formula: gain ~ 1 + stud_pre_cent + used_las + collabnla + FMCE + (1 |  
##     crse_id)
##    Data: SampleData
## 
## REML criterion at convergence: 51891.4
## 
## Scaled residuals: 
##     Min      1Q  Median      3Q     Max 
## -4.4172 -0.6589 -0.0097  0.6495  3.7714 
## 
## Random effects:
##  Groups   Name        Variance Std.Dev.
##  crse_id  (Intercept)  61.59    7.848  
##  Residual             341.47   18.479  
## Number of obs: 5959, groups:  crse_id, 112
## 
## Fixed effects:
##               Estimate Std. Error t value
## (Intercept)   14.18570    2.07169   6.847
## stud_pre_cent -0.45137    0.01305 -34.595
## used_las       5.92076    2.33039   2.541
## collabnla      1.66011    2.77476   0.598
## FMCE           0.25675    2.13447   0.120
## 
## Correlation of Fixed Effects:
##             (Intr) std_p_ usd_ls cllbnl
## stud_pr_cnt  0.000                     
## used_las    -0.876  0.000              
## collabnla   -0.741  0.000  0.669       
## FMCE        -0.060  0.000 -0.159 -0.048
\end{verbatim}

\begin{Shaded}
\begin{Highlighting}[]
\CommentTok{#MLR models}
\KeywordTok{summary}\NormalTok{(MLR1)}
\end{Highlighting}
\end{Shaded}

\begin{verbatim}
## 
## Call:
## lm(formula = mlr_mod1, data = SampleData)
## 
## Residuals:
##     Min      1Q  Median      3Q     Max 
## -96.028 -14.892  -1.801  13.901  76.291 
## 
## Coefficients:
##             Estimate Std. Error t value Pr(>|t|)    
## (Intercept)  19.4323     0.2806   69.24   <2e-16 ***
## ---
## Signif. codes:  0 '***' 0.001 '**' 0.01 '*' 0.05 '.' 0.1 ' ' 1
## 
## Residual standard error: 21.66 on 5958 degrees of freedom
\end{verbatim}

\begin{Shaded}
\begin{Highlighting}[]
\KeywordTok{summary}\NormalTok{(MLR2)}
\end{Highlighting}
\end{Shaded}

\begin{verbatim}
## 
## Call:
## lm(formula = mlr_mod2, data = SampleData)
## 
## Residuals:
##     Min      1Q  Median      3Q     Max 
## -96.143 -14.243  -1.057  13.786  78.964 
## 
## Coefficients:
##             Estimate Std. Error t value Pr(>|t|)    
## (Intercept)  12.1419     0.7609   15.96   <2e-16 ***
## used_las      7.4055     0.8311    8.91   <2e-16 ***
## collabnla    12.2481     1.0039   12.20   <2e-16 ***
## ---
## Signif. codes:  0 '***' 0.001 '**' 0.01 '*' 0.05 '.' 0.1 ' ' 1
## 
## Residual standard error: 21.4 on 5956 degrees of freedom
## Multiple R-squared:  0.02444,    Adjusted R-squared:  0.02411 
## F-statistic: 74.61 on 2 and 5956 DF,  p-value: < 2.2e-16
\end{verbatim}

\begin{Shaded}
\begin{Highlighting}[]
\KeywordTok{summary}\NormalTok{(MLR3)}
\end{Highlighting}
\end{Shaded}

\begin{verbatim}
## 
## Call:
## lm(formula = mlr_mod3, data = SampleData)
## 
## Residuals:
##     Min      1Q  Median      3Q     Max 
## -79.541 -13.766  -0.706  13.506  65.319 
## 
## Coefficients:
##               Estimate Std. Error t value Pr(>|t|)    
## (Intercept)   12.14188    0.70168  17.304   <2e-16 ***
## stud_pre_cent -0.45137    0.01393 -32.394   <2e-16 ***
## used_las       7.40550    0.76639   9.663   <2e-16 ***
## collabnla     12.24810    0.92576  13.230   <2e-16 ***
## ---
## Signif. codes:  0 '***' 0.001 '**' 0.01 '*' 0.05 '.' 0.1 ' ' 1
## 
## Residual standard error: 19.73 on 5955 degrees of freedom
## Multiple R-squared:  0.1706, Adjusted R-squared:  0.1702 
## F-statistic: 408.3 on 3 and 5955 DF,  p-value: < 2.2e-16
\end{verbatim}

\begin{Shaded}
\begin{Highlighting}[]
\KeywordTok{summary}\NormalTok{(MLR4)}
\end{Highlighting}
\end{Shaded}

\begin{verbatim}
## 
## Call:
## lm(formula = mlr_mod4, data = SampleData)
## 
## Residuals:
##     Min      1Q  Median      3Q     Max 
## -79.541 -13.766  -0.706  13.506  65.319 
## 
## Coefficients:
##               Estimate Std. Error t value Pr(>|t|)    
## (Intercept)   12.14188    0.70168  17.304   <2e-16 ***
## stud_pre_cent -0.45137    0.01393 -32.394   <2e-16 ***
## used_las       7.40550    0.76639   9.663   <2e-16 ***
## collabnla     12.24810    0.92576  13.230   <2e-16 ***
## ---
## Signif. codes:  0 '***' 0.001 '**' 0.01 '*' 0.05 '.' 0.1 ' ' 1
## 
## Residual standard error: 19.73 on 5955 degrees of freedom
## Multiple R-squared:  0.1706, Adjusted R-squared:  0.1702 
## F-statistic: 408.3 on 3 and 5955 DF,  p-value: < 2.2e-16
\end{verbatim}

\begin{Shaded}
\begin{Highlighting}[]
\KeywordTok{summary}\NormalTok{(MLR5)}
\end{Highlighting}
\end{Shaded}

\begin{verbatim}
## 
## Call:
## lm(formula = mlr_mod5, data = SampleData)
## 
## Residuals:
##     Min      1Q  Median      3Q     Max 
## -80.173 -13.674  -0.841  13.577  66.268 
## 
## Coefficients:
##                Estimate Std. Error t value Pr(>|t|)    
## (Intercept)    12.62950    0.73975  17.073   <2e-16 ***
## stud_pre_cent  -0.45137    0.01393 -32.403   <2e-16 ***
## used_las        6.95050    0.79689   8.722   <2e-16 ***
## collabnla      11.51532    0.99049  11.626   <2e-16 ***
## class_pre_cent  0.06258    0.03014   2.077   0.0379 *  
## ---
## Signif. codes:  0 '***' 0.001 '**' 0.01 '*' 0.05 '.' 0.1 ' ' 1
## 
## Residual standard error: 19.73 on 5954 degrees of freedom
## Multiple R-squared:  0.1712, Adjusted R-squared:  0.1706 
## F-statistic: 307.5 on 4 and 5954 DF,  p-value: < 2.2e-16
\end{verbatim}

\begin{Shaded}
\begin{Highlighting}[]
\KeywordTok{summary}\NormalTok{(MLR6)}
\end{Highlighting}
\end{Shaded}

\begin{verbatim}
## 
## Call:
## lm(formula = mlr_mod6, data = SampleData)
## 
## Residuals:
##     Min      1Q  Median      3Q     Max 
## -79.087 -13.721  -0.768  13.573  65.773 
## 
## Coefficients:
##               Estimate Std. Error t value Pr(>|t|)    
## (Intercept)   12.17253    0.70200  17.340   <2e-16 ***
## stud_pre_cent -0.45137    0.01393 -32.396   <2e-16 ***
## used_las       7.72885    0.80275   9.628   <2e-16 ***
## collabnla     12.25982    0.92573  13.243   <2e-16 ***
## FMCE          -0.80815    0.59738  -1.353    0.176    
## ---
## Signif. codes:  0 '***' 0.001 '**' 0.01 '*' 0.05 '.' 0.1 ' ' 1
## 
## Residual standard error: 19.73 on 5954 degrees of freedom
## Multiple R-squared:  0.1709, Adjusted R-squared:  0.1703 
## F-statistic: 306.7 on 4 and 5954 DF,  p-value: < 2.2e-16
\end{verbatim}

Assumption checking

\begin{Shaded}
\begin{Highlighting}[]
\CommentTok{#linearity: Shouldn't see a pattern}
\KeywordTok{plot}\NormalTok{(HLM3)}
\end{Highlighting}
\end{Shaded}

\includegraphics{Supplemental_Material_files/figure-latex/unnamed-chunk-8-1.pdf}

\begin{Shaded}
\begin{Highlighting}[]
\CommentTok{#quantitative homogeneity of variance}
\NormalTok{SampleData}\OperatorTok{$}\NormalTok{Model.F.Res<-}\StringTok{ }\KeywordTok{residuals}\NormalTok{(HLM3) }\CommentTok{#extracts the residuals and places them in a new column in our original data table}
\NormalTok{SampleData}\OperatorTok{$}\NormalTok{Abs.Model.F.Res <-}\KeywordTok{abs}\NormalTok{(SampleData}\OperatorTok{$}\NormalTok{Model.F.Res) }\CommentTok{#creates a new column with the absolute value of the}

\NormalTok{Levene.Model.F <-}\StringTok{ }\KeywordTok{lm}\NormalTok{(Model.F.Res }\OperatorTok{~}\StringTok{ }\NormalTok{crse_id, }\DataTypeTok{data=}\NormalTok{SampleData) }\CommentTok{#ANOVA of the residuals}
\KeywordTok{anova}\NormalTok{(Levene.Model.F) }\CommentTok{#displays the results: want a p>0.05}
\end{Highlighting}
\end{Shaded}

\begin{verbatim}
## Analysis of Variance Table
## 
## Response: Model.F.Res
##             Df  Sum Sq Mean Sq F value Pr(>F)
## crse_id      1      47   46.98  0.1397 0.7085
## Residuals 5957 2002574  336.17
\end{verbatim}

\begin{Shaded}
\begin{Highlighting}[]
\CommentTok{#visual homogeneity of variance}
\KeywordTok{boxplot}\NormalTok{(SampleData}\OperatorTok{$}\NormalTok{Model.F.Res }\OperatorTok{~}\StringTok{ }\NormalTok{SampleData}\OperatorTok{$}\NormalTok{crse_id)}
\end{Highlighting}
\end{Shaded}

\includegraphics{Supplemental_Material_files/figure-latex/unnamed-chunk-8-2.pdf}

\begin{Shaded}
\begin{Highlighting}[]
\CommentTok{#Assumption of Normality or residuals: want points to be near the line}
\KeywordTok{qqmath}\NormalTok{(HLM3)}
\end{Highlighting}
\end{Shaded}

\includegraphics{Supplemental_Material_files/figure-latex/unnamed-chunk-8-3.pdf}

Creating groups for final model

\begin{Shaded}
\begin{Highlighting}[]
\NormalTok{Trad =}\StringTok{ }\KeywordTok{c}\NormalTok{(}\DecValTok{1}\NormalTok{,}\DecValTok{0}\NormalTok{,}\DecValTok{0}\NormalTok{,}\DecValTok{0}\NormalTok{)}
\NormalTok{Collab =}\StringTok{ }\KeywordTok{c}\NormalTok{(}\DecValTok{1}\NormalTok{,}\DecValTok{0}\NormalTok{,}\DecValTok{0}\NormalTok{,}\DecValTok{1}\NormalTok{)}
\NormalTok{LA =}\StringTok{ }\KeywordTok{c}\NormalTok{(}\DecValTok{1}\NormalTok{,}\DecValTok{0}\NormalTok{,}\DecValTok{1}\NormalTok{,}\DecValTok{0}\NormalTok{)}


\NormalTok{HLM_preds <-}\StringTok{ }\KeywordTok{rbind}\NormalTok{( }\StringTok{'Lecture'}\NormalTok{=Trad, }\StringTok{'Collaborative'}\NormalTok{=Collab, }
                          \StringTok{'LAs'}\NormalTok{=LA)}
\end{Highlighting}
\end{Shaded}

\begin{Shaded}
\begin{Highlighting}[]
\CommentTok{# getting summary statistics from HLM model for plot}
\NormalTok{  sxp3 <-}\StringTok{ }\KeywordTok{summary}\NormalTok{(}\KeywordTok{glht}\NormalTok{(HLM3, }\DataTypeTok{linfct=}\NormalTok{HLM_preds)) }\CommentTok{#getting the summary from the HLM models}
\NormalTok{  get.est<-}\StringTok{ }\KeywordTok{data.frame}\NormalTok{(}\DataTypeTok{analysis =} \KeywordTok{c}\NormalTok{(}\StringTok{"HLM"}\NormalTok{,}\StringTok{"HLM"}\NormalTok{,}\StringTok{"HLM"}\NormalTok{), }\CommentTok{#simplifying that summary for the plots}
               \DataTypeTok{group=}\KeywordTok{rownames}\NormalTok{(sxp3}\OperatorTok{$}\NormalTok{linfct),}
             \DataTypeTok{coeff =}\NormalTok{ sxp3}\OperatorTok{$}\NormalTok{test}\OperatorTok{$}\NormalTok{coefficients, }
             \DataTypeTok{se =}\NormalTok{ sxp3}\OperatorTok{$}\NormalTok{test}\OperatorTok{$}\NormalTok{sigma)}

  
\CommentTok{# getting summary statistics from MLR model for plot}
\NormalTok{  sxp3 <-}\StringTok{ }\KeywordTok{summary}\NormalTok{(}\KeywordTok{glht}\NormalTok{(MLR3, }\DataTypeTok{linfct=}\NormalTok{HLM_preds)) }
\NormalTok{  temp<-}\StringTok{ }\KeywordTok{data.frame}\NormalTok{(}\DataTypeTok{analysis =} \KeywordTok{c}\NormalTok{(}\StringTok{"MLR"}\NormalTok{,}\StringTok{"MLR"}\NormalTok{,}\StringTok{"MLR"}\NormalTok{),}
                    \DataTypeTok{group=}\KeywordTok{rownames}\NormalTok{(sxp3}\OperatorTok{$}\NormalTok{linfct),}
             \DataTypeTok{coeff =}\NormalTok{ sxp3}\OperatorTok{$}\NormalTok{test}\OperatorTok{$}\NormalTok{coefficients, }
             \DataTypeTok{se =}\NormalTok{ sxp3}\OperatorTok{$}\NormalTok{test}\OperatorTok{$}\NormalTok{sigma)}

\CommentTok{#combine MLR and HLM summaries for plot}
\NormalTok{  get.est <-}\StringTok{ }\KeywordTok{bind_rows}\NormalTok{(get.est,temp)}
\end{Highlighting}
\end{Shaded}

Graph of Model 3 predicted values with error bars representing 1
standard error

\begin{Shaded}
\begin{Highlighting}[]
\KeywordTok{ggplot}\NormalTok{(get.est, }\KeywordTok{aes}\NormalTok{(}\DataTypeTok{y=}\NormalTok{coeff, }\DataTypeTok{fill=}\NormalTok{analysis, }\DataTypeTok{x=}\NormalTok{group )) }\OperatorTok{+}
\StringTok{  }\KeywordTok{geom_bar}\NormalTok{(}\DataTypeTok{stat=}\StringTok{"identity"}\NormalTok{, }\DataTypeTok{position =} \KeywordTok{position_dodge}\NormalTok{(}\DataTypeTok{width=}\FloatTok{0.9}\NormalTok{)) }\OperatorTok{+}\StringTok{ }
\StringTok{  }\KeywordTok{geom_errorbar}\NormalTok{(}\KeywordTok{aes}\NormalTok{(}\DataTypeTok{ymax=}\NormalTok{coeff}\OperatorTok{+}\NormalTok{se, }\DataTypeTok{ymin=}\NormalTok{coeff}\OperatorTok{-}\NormalTok{se), }\DataTypeTok{position=}\KeywordTok{position_dodge}\NormalTok{(}\FloatTok{0.9}\NormalTok{), }\DataTypeTok{width=}\FloatTok{0.5}\NormalTok{)  }\OperatorTok{+}\StringTok{ }
\StringTok{  }\KeywordTok{scale_fill_brewer}\NormalTok{(}\DataTypeTok{palette=}\StringTok{"Paired"}\NormalTok{)}\OperatorTok{+}
\StringTok{  }\KeywordTok{ylab}\NormalTok{(}\StringTok{"Gain (% points)"}\NormalTok{) }\OperatorTok{+}\StringTok{ }
\StringTok{  }\KeywordTok{xlab}\NormalTok{(}\StringTok{""}\NormalTok{) }\OperatorTok{+}
\StringTok{  }\KeywordTok{theme}\NormalTok{(}\DataTypeTok{legend.position =} \StringTok{"bottom"}\NormalTok{, }\DataTypeTok{legend.direction =} \StringTok{"horizontal"}\NormalTok{, }\DataTypeTok{legend.title =} \KeywordTok{element_blank}\NormalTok{()) }
\end{Highlighting}
\end{Shaded}

\includegraphics{Supplemental_Material_files/figure-latex/unnamed-chunk-11-1.pdf}


\end{document}
